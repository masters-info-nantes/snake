\documentclass[12pt,a4paper]{article}

\usepackage[utf8]{inputenc}
\usepackage[T1]{fontenc}
\usepackage{lmodern}

\usepackage{graphicx}
\usepackage{microtype}
\usepackage{hyperref} 
\usepackage[frenchb]{babel}

\usepackage{listings}	   
\lstdefinestyle{customstyle}{
    basicstyle=\footnotesize,
    breakatwhitespace=false,         
    breaklines=true,                 
    captionpos=b,                    
    keepspaces=true,                                                                                       
    tabsize=4,
    frame=single,
    moredelim=[is][\underbar]{_}{_}
}
\lstset{style=customstyle}

\title{MGS plugin framework manual}
\author{Antoine Forgerou \and David Drevet \and Jérémy Bardon}
\date{}
	
\begin{document}
	\renewcommand{\contentsname}{Sommaire}
	\renewcommand{\arraystretch}{1.8}
	\maketitle	
	
	\vspace{0.80cm}
	\tableofcontents	

	\thispagestyle{empty}	
	\setcounter{page}{0}
	\newpage
	
\begin{abstract}
Le framework MGS permet de gérer de manière transparente la modularité des 
applications basées sur l'utilisation de plugins. Plutot que d'obliger ces 
utilisateurs à remplir de long fichiers de configurations, nous avons choisi 
le plus possible de nous concentrer sur une architecture de fichier la moins
contraignante possible.
\end{abstract}

\section{Organisation du framework}
\subsection{Configuration}
Il existe un seul fichier de configuration qui permet d'ajuster le fonctionnement 
du framework. Ce fichier se nomme \emph{settings.txt} et se trouve dans le 
répertoire \emph{ressources} au même niveau que les fichier sources internes au 
framework.

\begin{table}[h]
\centering
	\begin{tabular}{lp{9cm}}
		pluginspath & Chemin vers le dossier qui contient tout les plugins\\
					 
		startplugin & Nom du plugin\footnote{Il s'agit du nom du dossier du 
		plugin} à charger au démarrage. Obligatoirement runnable.\\					 
	\end{tabular}	
\caption{Paramètres du fichier settings.txt}
\end{table}
	
Ce répertoire contient également le fichier \emph{jar} qui rassemble les plugins.
\subsection{Plugins}
Le dossier plugin renseigné par \emph{pluginpath} est le répertoire qui contient 
tout les plugins que le framework pourra charger. On distingue deux types de 
plugins:

\begin{description}
	\item[Runnable] Plugin principal qui peut etre démarré par le framework s'il 
	renseigné comme \emph{startplugin} dans le fichier de configuration du 
	framework.
	
	\item[Classic] Plugin annexe qui fourni des classes respectant les interfaces 
	définies par un ou plusieurs plugin(s) runnable qui l'utiliseront.
\end{description}

\subsubsection{Fichier de configuration}
Chaque plugin -- quelque soit son type -- doit respecter une certaine architecture 
en termes d'organisation de son dossier. En effet, un plugin doit obligatoirement 
fournir un fichier \emph{plugin.txt} qui le décrit et donne des informations sur 
comment il peut etre utilisé.

\begin{table}[h]\label{tab:plugintxt}
\centering
	\begin{tabular}{lp{10cm}}
		runnable & Indique si le plugin est un plugin runnable ou classic. 
		Peut être égal à \emph{0/1} ou \emph{true/false}.\\				
		
		category & Classe ou interface que respecte le plugin. Pour un plugin 
		runnable ce sera toujours \emph{fr.univnantes.snake.framwork.MGSApplication} 
		qui sera rempli automatiquement par le framework. Dans le cas d'un plugin 
		classique ce sera une interface définie par un plugin runnable.\\
		
		mainClass & Classe principale qui sera chargée par le framework. 
		Elle doit implémenter ou hériter de celle donnée par la categorie.\\
	\end{tabular}	
\caption{Paramètres d'un fichier plugin.txt}
\end{table}

\subsubsection{Détail sur les plugins runnable}\label{sss:DetailsRunnable}
Les plugins principaux sont capables de définir des interfaces afin de s'assurer 
que les plugins qu'ils utiliseront répondent à leurs besoins.
\\\\
Plutôt que de donner la liste des interfaces dans le fichier de configuration -- 
ce qui lourd -- nous avons choisi d'obliger l'utilisateur à placer ces interfaces 
dans un sous-répertoire \emph{interfaces}. Ceci est moins contraignant en plus du 
fait que des utilisateurs organisées les auraient de toutes façons rassemblés 
dans un package à part.

\section{Utilisation}
\subsection{Créer un plugin principal}
Pour commencer, il est nécessaire de vérifier où le framework va chercher les 
plugins en regardant le fichier \emph{settings.txt}. Dans notre exemple ce sera 
\emph{/home/user/plugins} mais il peut se trouver à n'importe quel endroit à 
condition d'avoir les droits de lecture et un chemin par trop exotique avec des 
espaces et des caractères accentués.
\\\\
Créons le plugin \emph{hello} qui aura pour but de dire bonjour dans différentes 
langues selon le plugin secondaire qui sera utilisé.

\begin{lstlisting}[language=bash,caption=Création du plugin hello]
$ mkdir /home/user/plugins/hello
$ cd /home/user/plugins/hello
$ touch plugins.txt Hello.java
\end{lstlisting}

Le fichier de configuration indique que le plugin \emph{hello} est un plugin 
runnable et que classe à lancer (qui implémente MGSApplication) est \emph{Hello}.

\lstset{language=bash,caption=Hello plugin configuration}
\lstinputlisting{ressources/plugin-hello.txt}

La liaison entre le framework et le plugin principal se fait à travers la relation 
d'héritage avec la classe MGSApplication. 
\\\\
En fait, nous avons fait ce choix car l'utilisation d'un singleton aurait pu 
donner accès au framework à n'importe quel plugin (même secondaire). De plus, si 
nous avions choisi de passer l'accès en paramètre de la fonction surchargée -- 
\emph{run} -- ceci aurait obligé l'utilisateur à ajouter un attribut pour stocker 
l'accès au framework.

\lstset{language=java,caption=Hello plugin main class}
\lstinputlisting{ressources/Hello.java}

La dernière chose à faire est de dire au framework de démarrer notre plugin au 
démarage. Pour cela, il faut renseigner le nom du plugin dans le paramètre 
\emph{startplugin} du fichier de configuration du framework.

\subsection{Définir des patrons de plugins}
Comme expliqué précédemment (voir \ref{sss:DetailsRunnable}), il est possible de 
définir des plugins secondaires qui seront utilisés par le plugin principal. 
Tout d'abord, il faut commencer par définir le périmètre fonctionnel des 
plugins secondaires que l'on veut à travers une interface.

\lstset{language=java,caption=Speak interface for Hello plugin}
\lstinputlisting{ressources/Speak.java}

La prochaine étape consiste à créer un nouveau plugin -- \emph{frenchspeak} par 
exemple -- qui aura pour catégorie \emph{Speak}.

\begin{lstlisting}[language=bash,caption=Création du plugin secondaire frenchspeak]
$ mkdir /home/user/plugins/frenchspeak
$ cd /home/user/plugins/frenchspeak
$ touch plugins.txt French.java
\end{lstlisting}

Ce nouveau plugin aura pour classe principale \emph{French} qui implémentera 
l'interface \emph{Speak} définie dans le plugin \emph{Hello} plus tôt.
\lstset{language=bash,caption=FrenchSpeak plugin configuration}
\lstinputlisting{ressources/plugin-frenchspeak.txt}


\lstset{language=java,caption=FrenchSpeak plugin main class}
\lstinputlisting{ressources/French.java}

\subsection{Donner les plugins au framework}
La dernière chose à faire et donner accès au framework à nos plugins. Pour cela 
il faut fournir le projet \emph{plugins} sour forme de \emph{jar} et le placer 
dans le répertoire \emph{platform/ressources}.
\\\\
Tout les plugins étant rassemblés dans le projet maven \emph{plugins}, voici 
la marche à suivre pour mettre à jour les mettre à jour.

\begin{lstlisting}[language=bash,caption=Mise à jour des plugins]
$ cd plugins
$ mvn compile
$ mvn package
$ cp target/snake-0.1.jar ../platform/ressources
\end{lstlisting}

\subsection{Appels au framework}
Etant donné qu'un plugin principal peut utiliser des plugins secondaires gérés 
par le framework, il est possible d'instancier des plugins secondaires à 
partir d'un plugin principal.
\\\\
La classe \emph{Hello} du plugin \emph{hello} doit implémenter la méthode 
\emph{run} qui prend en paramètre un objet de type \emph{AppContext} et c'est 
ce lien qui va permettre de faire des appels au framework. 
Cette classe possède deux attributs accessibles par le plugin (via des getters):
\emph{currentPlugin} et \emph{pluginLoader}. Le premier regroupe tout simplement 
la configuration du plugin courant -- hello -- mais le second donne accès au 
gestionnaire de plugins.

\begin{table}[h]
	\begin{tabular}{|l|}
		\hline
		\texttt{String getPluginsPath()}\\
		\hline
		Chemin vers le dossier où se trouvent les plugins\\
		\hline			

		\hline
		\texttt{Collection<Plugin> getClassicPlugins()}\\
		\hline
		Configuration de tout les plugins chargés par le framework (exepté les\\ 
		plugins	principaux).\\
		\hline
	
		\hline
		\texttt{List<Plugin> getClassicPluginsByCategory(String category)}\\
		\hline
		Tout les plugins secondaires ayant la catégorie donnée\\
		\hline

		\hline
		\texttt{Set<String> getClassicPluginsList()}\\
		\hline
		Nom de tout les plugins chargés par le framework (exepté les \\
		plugins	principaux).\\
		\hline
			
		\hline
		\texttt{Object loadPlugin(Plugin plugin) throws IOException}\\
		\hline
		Charge le plugin donné en paramètre. Cette fonction assure que le plugin\\
		donné est bien une sous-classe de sa catégorie.\\
		\hline
	\end{tabular}	
\caption{Méthodes du gestionnaire de plugins}
\end{table}

\section{Fonctionnement interne}
Lors d'une première phase, le framework va lire son fichier de configuration -- 
\emph{settings.txt} -- pour trouver le chemin vers le dossier des plugins. 
Ensuite, il va scanner les sous-dossiers, lire le fichier de configuration 
\emph{plugins.txt} et stocker la liste des plugins dont la configuration est valide.

\begin{quote}
	\textbf{Attention !} Le framework indique au démarrage pour chaque plugin 
	s'il a été chargé ou non. 
	\\\\
	Certaines propriétés du fichier \emph{plugin.txt} sont déduites par le 
	framework: \emph{runnable} est faux par défaut et ce n'est pas obligé de 
	donner la \emph{category} pour un plugin principal car c'est toujours la 
	même (voir \ref{tab:plugintxt}).
\end{quote}

Une fois cette étape de chargement de la liste des plugins, le framework va 
démarrer le plugin par défaut indiqué avec \emph{startplugin} dans son fichier 
de configuration.
\end{document}